\documentclass[nochap,palatino]{apuntes}

\usepackage{sagetex}

\begin{document}

\section{Análisis sistemático de una función}

\begin{problem}
Analiza sistemáticamente la siguiente función:

\begin{sagesilent}
f = (x^2-5*x+6)/(x-4)
\end{sagesilent}

\[f(x) = \sagestr{latex(f(x))}\]

\solution


\subsection{Asíntotas}
\subsubsection{Asíntotas verticales}

\begin{sagesilent}
den=f.denominator(normalize=False)
den0=solve(f.denominator(normalize=False) == 0,x,solution_dict=True)
def asintotesV():
    r=""
    k=0

    r+= "Soluciones: "
    for i in den0:
        r+="$x_"+str(k)+"="+str(i[x])+" $;\\quad"
        k+=1

    for x0 in den0:
        r+= "\n \\paragraph{Asintota en x ="
        r+= str(x0[x])+"}"
        ld=limit(f,x=x0[x],dir="plus")
        li=limit(f,x=x0[x],dir="minus")
        r+= "\n Calculamos \[\\lim_{x\\mapsto " + str(x0[x]) + "} " + latex(f(x)) + " = \\left\\{\\begin{array}{cc}\\text{Por la derecha: } & x\\mapsto" + latex(ld) + " \\\\ \\text{Por la izquierda: } & x\\mapsto" + latex(li) + "\\end{array}\\right.\] "
        if (abs(li) == Infinity and abs(ld) == Infinity):
            r+= "\n\t     La recta x =" + str(x0[x]) + " es una asíntota vertical de f(x)"
        else:
            r+= "\n No hay asíntota vertical en x="+str(x0[x])+"."
    return r
\end{sagesilent}

Los posibles puntos de asíntota son los que anulan el denominador, para ello calculamos:
$ \sagestr{latex(den)} = 0 $.

\sagestr{asintotesV()}

\subsubsection{Asíntotas horizontales u oblícuas}

Para ello calculamos el límite tanto en $+\infty$ como en $-\infty$:

\[\lim_{x\mapsto \pm\infty} \sagestr{latex(f(x))} \]

\begin{sagesilent}
def asintotesHO():
    r=""
    for infty in [Infinity, -Infinity]:
        r+= "\\paragraph{En $"+latex(infty)+"$:}"
        ld=limit(f, x=infty)  

        r+= "\[\\lim_{x\\mapsto "+latex(infty)+"}" + latex(f(x)) + "="+latex(ld)+"\]"    

        if (not (abs(ld) == Infinity)): # HORIZONTAL
            r+= "En $"+latex(infty)
            r+= "$ f(x) tiene una asíntota horizontal en y = $"+latex(ld)+"$"

        else: # OBLICUA
            r+= "Podemos tener una asíntota oblícua. Para comprobarlo calculamos: "
            r+= "\[m=\\lim_{x\\mapsto "+latex(infty)+"} \\frac{"+latex(f(x))+"}{x}=\\lim_{x\\mapsto "+latex(infty)+"} "+latex(f(x)/x)+"\]"
            _m=limit(f/x,x=infty)
            m=_m
            if (abs(m) != Infinity):
                r+= "        En este caso+ ese límite tenemos m="+latex(m)+" por lo que sí hay asíntota oblícua. Calculamos n:"
                r+= "    \[n=\\lim_{x\\mapsto "+latex(infty)+"}\\left("+latex(f(x))+"-"+latex(m)+"x \\right) = \\lim_{x\\mapsto "+latex(infty)+"}"+latex(f(x)-m*x)+"\]"
                _n = limit(f-m*x,x=infty)
                n=_n
                y=m*x+n
                r+= "En $"+latex(infty)+"$ f(x) tiene una asíntota oblícua en y="+latex(y)
            else:
                r+= "En este caso, ese límite tenemos m="+latex(m)+" por lo que no hay asíntota oblícua."

    return r

\end{sagesilent}


\sagestr{asintotesHO()}



\end{problem}

\end{document}