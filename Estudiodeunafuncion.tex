\documentclass[nochap,palatino]{apuntes}

\usepackage{sagetex}

\begin{document}

\section{Análisis sistemático de una función}

\begin{problem}
Analiza sistemáticamente la siguiente función:

\begin{sagesilent}
f = (x^2-5*x+6)/(x^2-4)
\end{sagesilent}

\[f(x) = \sagestr{latex(f(x))}\]

\solution


\subsection{Asíntotas}
\subsubsection{Asíntotas verticales}

\begin{sagesilent}
den=f.denominator(normalize=False)
den0=solve(f.denominator(normalize=False) == 0,x,solution_dict=True)
def asintotesV():
    r=""
    k=0

    r+= "Soluciones: "
    for i in den0:
    	r+="$x_"+str(k)+"="+str(i[x])+" $;\\quad"
    	k+=1

    for x0 in den0:
        r+= "\n \\paragraph{Asintota en x ="
        r+= str(x0[x])+"}"
        ld=limit(f,x=x0[x],dir="plus")
        li=limit(f,x=x0[x],dir="minus")
        r+= "\n Calculamos \[\\lim_{x\\mapsto " + str(x0[x]) + "} " + latex(f(x)) + " = \\left\\{\\begin{array}{cc}\\text{Por la derecha: } & x\\mapsto" + latex(ld) + " \\\\ \\text{Por la izquierda: } & x\\mapsto" + latex(li) + "\\end{array}\\right.\] "
        if (abs(li) == Infinity and abs(ld) == Infinity):
            r+= "\n\t     La recta x =" + str(x0[x]) + " es una asíntota vertical de f(x)"
        else:
        	r+= "\n No hay asíntota vertical en x="+str(x0[x])+"."
    return r
\end{sagesilent}

Los posibles puntos de asíntota son los que anulan el denominador, para ello calculamos:
$ \sagestr{latex(den)} = 0 $.

\sagestr{asintotesV()}






\end{problem}

\end{document}